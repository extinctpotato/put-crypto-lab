\documentclass[12pt]{article}
\usepackage[a4paper, total={6in, 8in}]{geometry}
\usepackage[english,polish]{babel}
\usepackage[T1]{fontenc}
\usepackage{minted}
\usepackage{listings}
\usepackage{geometry}
\setlength{\parindent}{0pt}

\title{Podstawy kryptografii \\ \large Ciągi losowe, szyfrator strumieniowy}
\author{Adam Olech}

\begin{document}
\maketitle

\tableofcontents
\newpage

\section{Implementacja generatora ciągów binarnych BBS}

Na wejście funkcji generującej losowy ciąg znaków metodą \textbf{Blum-Blum-Shub}
podajemy dużą liczbę Bluma \textit{n} 
(uzyskaną poprzez iloczyn dwóch przystających do siebie liczb pierwszych),
liczbę naturalną \textit{r} denotującą długość ciągu
oraz losowo wybraną liczbę naturalną \textit{a}.

\begin{listing}[H]
	\inputminted[firstline=21,lastline=32]{python}{../bbs_generator.py}
	\caption{Kod generatora}
\end{listing}

Losowa liczba \textit{n} generowana jest w pętli przy użyciu metody \textit{random.randint}
tak długo, aż otrzymana liczba spełni warunek $GCD(a,n) = 1$.

\begin{listing}[H]
	\inputminted[firstline=14,lastline=19]{python}{../bbs_generator.py}
	\caption{Generacja losowej liczby}
\end{listing}

\newpage

\section{Cztery testy statystyczne według FIPS 140-2}

\subsection{Szczegóły implementacyjne}

Testy zostały zaimplementowane jako metody klasy, która jako argument inicjalizujący przyjmuje
tablicę bitów (\textit{list of 0-1 integers}).

\begin{listing}[H]
	\inputminted[firstline=4,lastline=9]{python}{../fips.py}
	\caption{Klasa FIPS}
\end{listing}

Jako konwencję przyjęto, że nazwy metod prywatnych (nie będące testami) zaczynają się od dwóch znaków podkreślenia,
a nazwy funkcji testujących kończą się wyrazem \lstinline{_test}.

\subsubsection{Test pojedynczych bitów}

\begin{listing}[H]
	\inputminted[firstline=34,lastline=37]{python}{../fips.py}
	\caption{Metoda klasy}
\end{listing}

\subsubsection{Test serii}

\begin{listing}[H]
	\inputminted[firstline=39,lastline=56]{python}{../fips.py}
	\caption{Metoda klasy}
\end{listing}

\subsubsection{Test długiej serii}

\begin{listing}[H]
	\inputminted[firstline=58,lastline=65]{python}{../fips.py}
	\caption{Metoda klasy}
\end{listing}

\subsubsection{Test pokerowy}

\begin{listing}[H]
	\inputminted[firstline=67,lastline=79]{python}{../fips.py}
	\caption{Metoda klasy}
\end{listing}

\subsection{Uruchomienie testów na 3 losowych ciągach o długości 200000 bitów}

\inputminted{text}{1-test-1.txt}
\inputminted{text}{1-test-2.txt}
\inputminted{text}{1-test-3.txt}

\subsection{Własności statystyczne ciągu o długości 1 000 000 bitów wg. pakietu NIST}

\inputminted[fontsize=\footnotesize]{text}{1-1mil-nist.txt}

\newpage

\section{Szyfrator strumieniowy}

\subsection{Zasada działania}

W programie zaimplementowano szyfrator strumieniowy wykorzystujący generator BBS
w celu wygenerowania klucza szyfrującego.
Klucz szyfrujący o długości pliku wejściowego 
generowany jest dla każdego pliku przed rozpoczęciem przetwarzania.

Istotą działania szyfratora strumieniowego jest operacja logiczna XOR.
Uzyskanie szyfrogramu polega na obliczeniu logicznej różnicy między ciągiem jawnym a kluczem,
a odszyfrowanie wiadomości odbywa się przez analogiczną operację na szyfrogramie i kluczu.

\begin{equation}
	c = m \oplus k 
\end{equation}

\begin{equation}
	m = c \oplus k
\end{equation}

Poniższa funkcja jest implementacją funkcji dokonującej wyżej opisanej operacji
na tabeli zerojedynkowych liczb całkowitoliczbowych reprezentujących stan logiczny.

\begin{listing}[H]
	\inputminted[firstline=49,lastline=50]{python}{../bbs_generator.py}
	\caption{Generacja tablicy licz całkowitoliczbowych}
\end{listing}

\subsection{Własności statystyczne przykładowego szyfrogramu}

Na potrzeby tego testu zaszyfrowano staroangielski epicki poemat heoriczny \textit{Beowulf}
pobrany ze strony projektu \textbf{The Project Gutenberg} oraz usunięto z niego znaki
których nie da się zakodować przy użyciu standardu ASCII.

W wyniku szyfrowania powstały 3 pliki:
\begin{itemize}
	\item \textit{beowulf-ascii.txt.bbs} -- szyfrogram w postaci zer i jedynek w ASCII
	\item \textit{beowulf-ascii.txt.bbs\_key} -- klucz w tej samej postaci jak powyżej 
	\item \textit{beowulf-ascii.txt.bin\_int} -- wiadomość jawna w tej samej postaci jak powyżej
\end{itemize}

Szyfrogram przechodzi zaimplementowane testy statystyczne:
\inputminted{text}{1-test-cipher.txt}

\end{document}

\documentclass[12pt]{article}
\usepackage[a4paper, total={6in, 8in}]{geometry}
\usepackage[english,polish]{babel}
\usepackage[T1]{fontenc}
\usepackage{minted}
\usepackage{listings}
\usepackage{geometry}

\title{Podstawy kryptografi \\ \large Ciągi losowe, szyfrator strumieniowy}
\author{Adam Olech}

\begin{document}
\maketitle

\tableofcontents
\newpage

\section{Implementacja generatora ciągów binarnych BBS}

Na wejście funkcji generującej losowy ciąg znaków metodą \textbf{Blum-Blum-Shub}
podajemy dużą liczbę Bluma \textit{n} 
(uzyskaną poprzez iloczyn dwóch przystających do siebie liczb pierwszych),
liczbę naturalną \textit{r} denotującą długość ciągu
oraz losowo wybraną liczbę naturalną \textit{a}.

\begin{listing}[H]
	\inputminted[firstline=21,lastline=32]{python}{../bbs_generator.py}
	\caption{Kod generatora}
\end{listing}

Losowa liczba \textit{n} generowana jest w pętli przy użyciu metody \textit{random.randint}
tak długo, aż otrzymana liczba spełni warunek $GCD(a,n) = 1$.

\begin{listing}[H]
	\inputminted[firstline=14,lastline=19]{python}{../bbs_generator.py}
	\caption{Generacja losowej liczby}
\end{listing}

\newpage

\section{Cztery testy statystyczne według FIPS 140-2}

\subsection{Szczegóły implementacyjne}

Testy zostały zaimplementowane jako metody klasy, która jako argument inicjalizujący przyjmuje
tablicę bitów (\textit{list of 0-1 integers}).

\begin{listing}[H]
	\inputminted[firstline=4,lastline=9]{python}{../fips.py}
	\caption{Klasa FIPS}
\end{listing}

Jako konwencję przyjęto, że nazwy metod prywatnych (nie będące testami) zaczynają się od dwóch znaków podkreślenia,
a nazwy funkcji testujących kończą się wyrazem \lstinline{_test}.

\subsubsection{Test pojedynczych bitów}

\begin{listing}[H]
	\inputminted[firstline=34,lastline=37]{python}{../fips.py}
	\caption{Metoda klasy}
\end{listing}

\subsubsection{Test serii}

\begin{listing}[H]
	\inputminted[firstline=39,lastline=56]{python}{../fips.py}
	\caption{Metoda klasy}
\end{listing}

\subsubsection{Test długiej serii}

\begin{listing}[H]
	\inputminted[firstline=58,lastline=65]{python}{../fips.py}
	\caption{Metoda klasy}
\end{listing}

\subsubsection{Test pokerowy}

\begin{listing}[H]
	\inputminted[firstline=67,lastline=79]{python}{../fips.py}
	\caption{Metoda klasy}
\end{listing}

\subsection{Uruchomienie testów na 3 losowych ciągach o długości 200000 bitów}

\inputminted{text}{1-test-1.txt}
\inputminted{text}{1-test-2.txt}
\inputminted{text}{1-test-3.txt}

\subsection{Własności statystyczne ciągu o długości 1 000 000 bitów wg. pakietu NIST}

\inputminted[fontsize=\footnotesize]{text}{1-1mil-nist.txt}

\end{document}


\documentclass[12pt]{article}
\usepackage[a4paper, total={6in, 8in}]{geometry}
\usepackage[english,polish]{babel}
\usepackage[T1]{fontenc}
\usepackage{minted}
\usepackage{geometry}
\geometry{left=2cm,right=2cm,top=1cm,bottom=1cm}

\title{Podstawy kryptografi \\ \large Ciągi losowe, szyfrator strumieniowy}
\author{Adam Olech}

\begin{document}
\maketitle

\begin{abstract}
Przedmiotem laboratorium są szyfry strumieniowe, generatory ciągów losowych oraz testowanie losowości.
Programy zaimplementowano w języku Python bez użycia bibliotek zewnętrznych.
\end{abstract}

\section{Implementacja generatora ciągów binarnych BBS}

Na wejście funkcji generującej losowy ciąg znaków metodą \textbf{Blum-Blum-Shub}
podajemy dużą liczbę Bluma \textit{n} 
(uzyskaną poprzez iloczyn dwóch przystających do siebie liczb pierwszych),
liczbę naturalną \textit{r} denotującą długość ciągu
oraz losowo wybraną liczbę naturalną \textit{a}.

\begin{listing}[H]
	\inputminted[firstline=21,lastline=32]{python}{../bbs_generator.py}
	\caption{Kod generatora}
\end{listing}

Losowa liczba \textif{n} generowana jest w pętli przy użyciu metody \textit{random.randint}
tak długo, aż otrzymana liczba spełni warunek $GCD(a,n) = 1$.

\begin{listing}[H]
	\inputminted[firstline=14,lastline=19]{python}{../bbs_generator.py}
	\caption{Generacja losowej liczby}
\end{listing}

\end{document}

\documentclass[12pt]{article}
\usepackage[a4paper, total={6in, 8in}]{geometry}
\usepackage[english,polish]{babel}
\usepackage[T1]{fontenc}
\usepackage{minted}
\usepackage{listings}
\usepackage{geometry}
\setlength{\parindent}{0pt}

% Dirty workaround for red boxes appearing around invalid syntax.
\newenvironment*{dummyenv}{}{}
\AtBeginEnvironment{dummyenv}{%
	  \renewcommand{\fcolorbox}[4][]{#4}}

\title{Podstawy kryptografii \\ \large Kryptografia symetryczna} 
\author{Adam Olech}

\begin{document}
\maketitle

\tableofcontents
\newpage

\section{Tryby pracy w PyCryptodome}

Analiza dostępnych trybów pracy została wykonana dla trzech plików o długościach
1, 10 i 100 megabajtów kolejno,
wygenerowanych w systemie z jądrem Linux przy użyciu programu \textit{dd},
podając wbudowany w system generator pseudolosowy dostępny jako urządzenie blokowe \textit{/dev/urandom}.

\subsection{Losowy plik o długości 1MB}

\begin{dummyenv}
	\inputminted[firstline=2,lastline=91]{yaml}{2-enc-dec-time.txt}
\end{dummyenv}

\subsection{Losowy plik o długości 10MB}

\begin{dummyenv}
	\inputminted[firstline=92,lastline=181]{yaml}{2-enc-dec-time.txt}
\end{dummyenv}

\subsection{Losowy plik o długości 100MB}

\begin{dummyenv}
	\inputminted[firstline=182,lastline=271]{yaml}{2-enc-dec-time.txt}
\end{dummyenv}

\subsection{Wnioski}

Wbrew intuicji, dla wszystkich plików operacje szyfracji i deszyfracji
zajmują najdłużej gdy wybrany jest tryb pracy ECB (który z perspektywy implementacji
wydaje się być najprostszy).

Rozmiar pliku wpływa na czas wykonywanych operacji (im większy plik, tym dłużej
zajmują operacje na nim).

Niektóre tryby pracy wykorzystują wektor początkowy (jest to wartość podawana
podczas wykonywania operacji na pierwszym bloku gdzie nie występuje blok poprzedni).

\end{document}

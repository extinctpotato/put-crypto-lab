\documentclass[12pt]{article}
\usepackage[a4paper, total={6in, 8in}]{geometry}
\usepackage[english,polish]{babel}
\usepackage[T1]{fontenc}
\usepackage{minted}
\usepackage{listings}
\usepackage{geometry}
\setlength{\parindent}{0pt}

% Dirty workaround for red boxes appearing around invalid syntax.
\newenvironment*{dummyenv}{}{}
\AtBeginEnvironment{dummyenv}{%
	  \renewcommand{\fcolorbox}[4][]{#4}}

\title{Podstawy kryptografii \\ \large Kryptografia symetryczna} 
\author{Adam Olech}

\begin{document}
\maketitle

\tableofcontents
\newpage

\section{Wstęp}

Informacje na temat wykonywanego ćwiczenia:

\begin{itemize}
	\item \textbf{Język programowania}: Python
	\item \textbf{Interpreter}: CPython (Python 3.9.9)
	\item \textbf{Algorytm szyfru blokowego}: AES
	\item \textbf{Biblioteka}: PyCryptodome
\end{itemize}

\section{Tryby pracy w PyCryptodome}

Analiza dostępnych trybów pracy została wykonana dla trzech plików o długościach
1, 10 i 100 megabajtów kolejno,
wygenerowanych w systemie z jądrem Linux przy użyciu programu \textit{dd},
podając wbudowany w system generator pseudolosowy dostępny jako urządzenie blokowe \textit{/dev/urandom}.

\subsection{Losowy plik o długości 1MB}

\begin{dummyenv}
	\inputminted[firstline=2,lastline=91]{yaml}{2-enc-dec-time.txt}
\end{dummyenv}

\subsection{Losowy plik o długości 10MB}

\begin{dummyenv}
	\inputminted[firstline=92,lastline=181]{yaml}{2-enc-dec-time.txt}
\end{dummyenv}

\subsection{Losowy plik o długości 100MB}

\begin{dummyenv}
	\inputminted[firstline=182,lastline=271]{yaml}{2-enc-dec-time.txt}
\end{dummyenv}

\subsection{Wnioski}

Wbrew intuicji, dla wszystkich plików operacje szyfracji i deszyfracji
zajmują najdłużej gdy wybrany jest tryb pracy ECB (który z perspektywy implementacji
wydaje się być najprostszy).

Rozmiar pliku wpływa na czas wykonywanych operacji (im większy plik, tym dłużej
zajmują operacje na nim).

Niektóre tryby pracy wykorzystują wektor początkowy (jest to wartość podawana
podczas wykonywania operacji na pierwszym bloku gdzie nie występuje blok poprzedni).

\newpage

\section{Implementacje trybów pracy}

Funkcje przedstawione poniżej są wykorzystywane do zaszyfrowania pojedynczego
bloku (\textit{input}) za pomocą klucza (\textit{key}).

\begin{listing}[H]
	\inputminted[firstline=9,lastline=13]{python}{../aes.py}
	\caption{Czarna skrzynka do szyfracji}
\end{listing}

\begin{listing}[H]
	\inputminted[firstline=15,lastline=19]{python}{../aes.py}
	\caption{Czarna skrzynka do deszyfracji}
\end{listing}

Aby podzielić listę na kawałki o jednakowej długości 
zaimplementowano stosowną funkcję.

\begin{listing}[H]
	\inputminted[firstline=21,lastline=24]{python}{../aes.py}
	\caption{Podział listy na kawałki}
\end{listing}

Dopełnienie wiadomości dokonywane jest przy użyciu funkcji
\textit{pad} oraz \textit{unpad} dostarczanych przez bibliotekę.

\begin{listing}[H]
	\inputminted[firstline=26,lastline=28]{python}{../aes.py}
	\caption{XORowanie list}
\end{listing}

\subsection{Electronic codebook (ECB)}

\begin{listing}[H]
	\inputminted[firstline=113,lastline=133]{python}{../aes.py}
	\caption{Szyfrowanie w trybie ECB}
\end{listing}

\subsection{Cipher block chaining (CBC)}

\begin{listing}[H]
	\inputminted[firstline=135,lastline=162]{python}{../aes.py}
	\caption{Szyfrowanie w trybie CBC}
\end{listing}

\subsection{Output feedback (OFB)}

\begin{listing}[H]
	\inputminted[firstline=164,lastline=185]{python}{../aes.py}
	\caption{Szyfrowanie w trybie CBC}
\end{listing}

\end{document}

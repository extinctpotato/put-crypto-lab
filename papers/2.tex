\documentclass[12pt]{article}
\usepackage[a4paper, total={6in, 8in}]{geometry}
\usepackage[english,polish]{babel}
\usepackage[T1]{fontenc}
\usepackage{minted}
\usepackage{listings}
\usepackage{geometry}
\setlength{\parindent}{0pt}

% Dirty workaround for red boxes appearing around invalid syntax.
\newenvironment*{dummyenv}{}{}
\AtBeginEnvironment{dummyenv}{%
	  \renewcommand{\fcolorbox}[4][]{#4}}

\title{Podstawy kryptografii \\ \large Kryptografia symetryczna} 
\author{Adam Olech}

\begin{document}
\maketitle

\tableofcontents
\newpage

\section{Wstęp}

Informacje na temat wykonywanego ćwiczenia:

\begin{itemize}
	\item \textbf{Język programowania}: Python
	\item \textbf{Interpreter}: CPython (Python 3.9.9)
	\item \textbf{Algorytm szyfru blokowego}: AES
	\item \textbf{Biblioteka}: PyCryptodome
\end{itemize}

\section{Tryby pracy w PyCryptodome}

Analiza dostępnych trybów pracy została wykonana dla trzech plików o długościach
1, 10 i 100 megabajtów kolejno,
wygenerowanych w systemie z jądrem Linux przy użyciu programu \textit{dd},
podając wbudowany w system generator pseudolosowy dostępny jako urządzenie blokowe \textit{/dev/urandom}.

\subsection{Losowy plik o długości 1MB}

\begin{dummyenv}
	\inputminted[firstline=2,lastline=91]{yaml}{2-enc-dec-time.txt}
\end{dummyenv}

\subsection{Losowy plik o długości 10MB}

\begin{dummyenv}
	\inputminted[firstline=92,lastline=181]{yaml}{2-enc-dec-time.txt}
\end{dummyenv}

\subsection{Losowy plik o długości 100MB}

\begin{dummyenv}
	\inputminted[firstline=182,lastline=271]{yaml}{2-enc-dec-time.txt}
\end{dummyenv}

\subsection{Wnioski}

Wbrew intuicji, dla wszystkich plików operacje szyfracji i deszyfracji
zajmują najdłużej gdy wybrany jest tryb pracy ECB (który z perspektywy implementacji
wydaje się być najprostszy).

Rozmiar pliku wpływa na czas wykonywanych operacji (im większy plik, tym dłużej
zajmują operacje na nim).

Niektóre tryby pracy wykorzystują wektor początkowy (jest to wartość podawana
podczas wykonywania operacji na pierwszym bloku gdzie nie występuje blok poprzedni).

\newpage

\section{Tryby pracy szyfrów blokowych}

Funkcje przedstawione poniżej są wykorzystywane do zaszyfrowania pojedynczego
bloku (\textit{input}) za pomocą klucza (\textit{key}).

\begin{listing}[H]
	\inputminted[firstline=9,lastline=13]{python}{../aes.py}
	\caption{Czarna skrzynka do szyfracji}
\end{listing}

\begin{listing}[H]
	\inputminted[firstline=15,lastline=19]{python}{../aes.py}
	\caption{Czarna skrzynka do deszyfracji}
\end{listing}

Aby podzielić listę na kawałki o jednakowej długości 
zaimplementowano stosowną funkcję.

\begin{listing}[H]
	\inputminted[firstline=21,lastline=24]{python}{../aes.py}
	\caption{Podział listy na kawałki}
\end{listing}

Dopełnienie wiadomości dokonywane jest przy użyciu funkcji
\textit{pad} oraz \textit{unpad} dostarczanych przez bibliotekę.

\begin{listing}[H]
	\inputminted[firstline=26,lastline=28]{python}{../aes.py}
	\caption{XORowanie list}
\end{listing}

\subsection{Implementacje}

\subsubsection{Electronic codebook (ECB)}

\begin{listing}[H]
	\inputminted[firstline=113,lastline=133]{python}{../aes.py}
	\caption{Szyfrowanie w trybie ECB}
\end{listing}

\subsubsection{Cipher block chaining (CBC)}

\begin{listing}[H]
	\inputminted[firstline=135,lastline=162]{python}{../aes.py}
	\caption{Szyfrowanie w trybie CBC}
\end{listing}

\subsubsection{Output feedback (OFB)}

\begin{listing}[H]
	\inputminted[firstline=164,lastline=185]{python}{../aes.py}
\end{listing}

\subsection{Weryfikacja poprawności działania}

Aby sprawdzić, czy tekst jawny po zaszfrowaniu i odszyfrowaniu jest niezmieniony,
można posłużyć się trzema funkcjami głownym entrypoincie
projektu (\textit{\detokenize{python3 cli.py aes_MODE_enc_test}}).

\begin{listing}[H]
	\inputminted[firstline=12,lastline=15]{python}{../aes_tests.py}
	\caption{Użyty tekst jawny}
\end{listing}

\section{Modyfikacja wiadomości tajnych}

\subsection{Opis modyfikacji}

Zaimplementowano funkcje wykonujące następujące operacje na blokach
wiadomości tajnej:

\begin{itemize}
	\item usunięcie bloku
	\item duplikacja bloku
	\item zamiana bloków miejscami (swap)
	\item zmiana jednego bajtu w bloku
	\item przetasowanie bajtów w bloku (shuffle)
	\item usunięcie fragmentu bloku
\end{itemize}

Wykorzystywane są trzy testowe ciągi tekstu jawnego.

\begin{listing}[H]
	\inputminted[firstline=20,lastline=24]{python}{../aes_tests.py}
	\caption{Testowe dane wejściowe}
\end{listing}

\newpage

\subsection{Implementacja manglera}

\inputminted[firstline=72,lastline=165]{python}{../aes_tests.py}

\subsection{Rezultat}

\subsubsection{ECB}

\begin{dummyenv}
	\inputminted[breaklines,firstline=1,lastline=51]{yaml}{2-mangled-ecb.txt}
\end{dummyenv}

\subsubsection{CBC}

\begin{dummyenv}
	\inputminted[breaklines,firstline=1,lastline=51]{yaml}{2-mangled-cbc.txt}
\end{dummyenv}

\subsubsection{OFB}

\begin{dummyenv}
	\inputminted[breaklines,firstline=1,lastline=51]{yaml}{2-mangled-ofb.txt}
\end{dummyenv}

\subsection{Wnioski}

Zamiana losowych bajtów dla wszystkich trybów z wyjątkiem OFB powoduje
znaczne zniekształcenie bloku.

Najprostszy z trybów, ECB, wydaje się być najmniej podatny na modyfikacje.

Wiadomości uzyskane w trybie CBC zdają się być bardziej zniekształcone niż te
pochodzące z trybu OFB.
\end{document}

\documentclass[12pt]{article}
\usepackage[a4paper, total={6in, 10in}]{geometry}
\usepackage[english,polish]{babel}
\usepackage[T1]{fontenc}
\usepackage{minted}
\usepackage{listings}
\usepackage{geometry}
\usepackage[hidelinks]{hyperref}
\hypersetup{colorlinks=false}
\setlength{\parindent}{0pt}

% Dirty workaround for red boxes appearing around invalid syntax.
\newenvironment*{dummyenv}{}{}
\AtBeginEnvironment{dummyenv}{%
	  \renewcommand{\fcolorbox}[4][]{#4}}

\title{Podstawy kryptografii \\ \large Kryptografia asymetryczna} 
\author{Adam Olech}

\begin{document}
\maketitle

\tableofcontents
\newpage

\section{Wstęp}

Informacje na temat wykonywanego ćwiczenia:

\begin{itemize}
	\item \textbf{Język programowania}: Python
	\item \textbf{Interpreter}: CPython (Python 3.9.9)
\end{itemize}

Kod dostępny jest w repozytorium pod adresem \url{https://github.com/extinctpotato/put-crypto-lab}

\section{Szyfrowanie przy użyciu algorytmu RSA}

\subsection{Generacja pary kluczy}

Aby wygenerować parę kluczy (publiczny i prywatny), 
przygotowano następujące liczby pierwsze:

\begin{itemize}
	\item \textbf{p}: 1013
	\item \textbf{q}: 2011
\end{itemize}

Wspomniane wyżej liczby zostały podane jako wejście na
funkcję implementującą algorytm generujący trzy składniki pary kluczy.

\begin{listing}[H]
	\inputminted[firstline=54,lastline=73]{python}{../rsa.py}
	\caption{Implementacja generatora pary kluczy}
\end{listing}

Funkcja przyjmuje na wejściu liczby $p$ i $q$, choć są to parametry opcjonalne.

W przypadku wywołania funkcji bez parametrów, zostaną użyte losowe liczby
pierwsze uzyskane przy pomocy funkcji 
\textit{randprime} z biblioteki \textbf{\href{https://www.sympy.org}{Sympy}}.

\end{document}

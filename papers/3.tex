\documentclass[12pt]{article}
\usepackage[a4paper, total={6in, 10in}]{geometry}
\usepackage[english,polish]{babel}
\usepackage[T1]{fontenc}
\usepackage{minted}
\usepackage{listings}
\usepackage{geometry}
\usepackage[hidelinks]{hyperref}
\hypersetup{colorlinks=false}
\setlength{\parindent}{0pt}

% Dirty workaround for red boxes appearing around invalid syntax.
\newenvironment*{dummyenv}{}{}
\AtBeginEnvironment{dummyenv}{%
	  \renewcommand{\fcolorbox}[4][]{#4}}

\title{Podstawy kryptografii \\ \large Kryptografia asymetryczna} 
\author{Adam Olech}

\begin{document}
\maketitle

\tableofcontents
\newpage

\section{Wstęp}

Informacje na temat wykonywanego ćwiczenia:

\begin{itemize}
	\item \textbf{Język programowania}: Python
	\item \textbf{Interpreter}: CPython (Python 3.9.9)
\end{itemize}

Kod dostępny jest w repozytorium pod adresem \url{https://github.com/extinctpotato/put-crypto-lab}

\section{Szyfrowanie przy użyciu algorytmu RSA}

\subsection{Generacja pary kluczy}

Aby wygenerować parę kluczy (publiczny i prywatny), 
przygotowano następujące liczby pierwsze:

\begin{itemize}
	\item \textbf{p}: 1013
	\item \textbf{q}: 2011
\end{itemize}

Wspomniane wyżej liczby zostały podane jako wejście na
funkcję implementującą algorytm generujący trzy składniki pary kluczy.

\begin{listing}[H]
	\inputminted[firstline=54,lastline=73]{python}{../rsa.py}
	\caption{Implementacja generatora pary kluczy}
\end{listing}

Funkcja przyjmuje na wejściu liczby $p$ i $q$, choć są to parametry opcjonalne.
\\

W przypadku wywołania funkcji bez parametrów, zostaną użyte losowe liczby
pierwsze uzyskane przy pomocy funkcji 
\textit{randprime} z biblioteki \textbf{\href{https://www.sympy.org}{Sympy}}.

\newpage

W wyniku działania funkcji uzyskano następujące liczby:

\begin{itemize}
	\item \textbf{e}: 1749151
	\item \textbf{d}: 11236
	\item \textbf{n}: 2037143
\end{itemize}

Para liczb $e$ oraz $n$ jest kluczem publicznym, a $d$ oraz $n$ kluczem prywatnym.

\subsubsection{Wyznaczenie wartości $e$}

Aby wyznaczyć wartość $e$, należy najpierw wyznaczyć tocjent (ang. \textit{totient}).
\\

Można to zrobić używając funkcji Carmichaela.
\begin{equation}
	\lambda (pq) = lcm(p-1, q-1)
\end{equation}

Alternatywą jest użycie funkcji Eulera
\begin{equation}
	\varphi (pq) = (p-q)(q-1)
\end{equation}

W przypadku tej pierwszej uzyskamy
najmniejszy wspólny mnożnik liczb $p-1$ oraz $q-1$,
\\

W przypadku, w którym nie dbamy, by była to liczba najmniejsza
\footnote{Mniejsza liczba zapewnia mniej kosztowną obliczeniowo operację
szyfrowania.},
dopuszczalne jest wykorzystanie metody Eulera (która da wielokrotność
wartości uzyskanej funkcją Carmichaela).
\\

Po wyznaczeniu tocjentu, wyznaczamy $e$ taką, która spełnia poniższe warunki:

\begin{itemize}
	\item liczba $e$ jest pierwsza.
	\item liczba $e$ jest większa od zera i mniejsza od tocjentu.
	\item największy wspólny dzielnik liczby $e$ i tocjentu nie wynosi $1$.
\end{itemize}

\newpage

\subsubsection{Wyznaczenie wartości $d$}

Aby wyznaczyć $d$, należy znaleźć taką liczbę,
która jest modularną odwrotnością multiplikatywną
reszty z dzielenia $e$ i tocjentu.

\begin{equation}
	d \equiv e^{{-1}} \pmod{\lambda (n)}
\end{equation}

Do tego celu można zastosować rozszerzony algorym Euklidesa.

\begin{listing}[H]
	\inputminted[firstline=32,lastline=42]{python}{../rsa.py}
	\caption{Implementacja rozszerzonego algorytmu Euklidesa}
\end{listing}

Ostatecznie:

\begin{listing}[H]
	\inputminted[firstline=28,lastline=30]{python}{../rsa.py}
	\caption{Wyliczenie modularnej odwrotności}
\end{listing}

\section{Szyfrowanie wiadomości}

\subsection{Opis metody}
Operacje szyfracji i deszyfracji kolejno odbywają się przez
wyliczenie reszty z dzielenia
podniesionego do potęgi liczby jawnej lub zaszyfrowanej do
liczby $e$ lub $d$ 
przez liczbę $d$.

\begin{listing}[H]
	\inputminted[firstline=75,lastline=79]{python}{../rsa.py}
	\caption{Implementacja RSA}
\end{listing}

\newpage

\subsection{Szyfrowanie ciągów}

Z racji, że algorytm RSA działa na liczbach całkowitoliczbowych,
szyfrowanie ciągów wymaga zamiany takiego ciągu na odpowiednią reprezentację
liczbową.
\\

Biblioteka standardowa w Pythonie 
pozwala na zamianę ciągu znaków na typ \textit{int}.

\begin{listing}[H]
	\inputminted[]{python}{3-bytes-example.py}
	\caption{Zamiana ciągu znaków na liczbę}
\end{listing}

Istotnym ograniczeniem jest maksymalna wartość liczby jawnej -- nie może
ona być większa od liczby $n$.
\\

Liczbę zatem należy zamienić
na wiele liczb w $n$-liczbowym systemie liczbowym.
Dzięki temu, maksymalną wartością każdej takiej liczby będzie $n$.

\begin{listing}[H]
	\inputminted[firstline=7,lastline=17]{python}{../rsa.py}
	\caption{Konwersja systemu liczbowego}
\end{listing}

Przy deszyfracji, operację należy odwrócić poprzez konwersję listy
liczb na system dziesiętny.

\begin{listing}[H]
	\inputminted[firstline=19,lastline=26]{python}{../rsa.py}
	\caption{Konwersja z systemu $n$-liczbowego na dziesiętny}
\end{listing}

\subsubsection{Przykład}

Zaszyfrowano 50 pierwszych znaków pewnego cytatu Arystotelesa.

\begin{listing}[H]
	\inputminted[firstline=12,lastline=15]{python}{../aes_tests.py}
	\caption{Tekst jawny}
\end{listing}

Funkcje opisane w poprzedniej sekcji zostały użyte w następujący sposób:

\begin{listing}[H]
	\inputminted[firstline=54,lastline=59]{python}{../cli.py}
	\caption{Tekst jawny}
\end{listing}

Po uruchomieniu funkcji otrzymujemy następujący rezultat:

\begin{listing}[H]
	\inputminted{yaml}{3-rsa-enc-test.txt}
	\caption{Wynik działania}
\end{listing}

\end{document}
